\newpage
\lipsum[24]

% You need inkscape for this
% inkscape needs to be in your PATH
\begin{figure}[H]
    \centering
    \includesvg[width=\textwidth]{Bilder/SurfPlot}
    \caption{Beispiel eines \texttt{Surface Plots} der drei Gleichungen des Gleichungssystems $A\vec{x} = \vec{b}$ nach $x_3$ aufgelöst}\label{fig:SurfPlot}
\end{figure}

In \autoref{fig:SurfPlot} findet sich ein Beispiel für eine \texttt{SVG}-Grafik. \lipsum[13]

\lipsum[32]

\begin{equation}
	E = \hbar \omega
	\label{eq:plancksrelation}
\end{equation}

\autoref{eq:plancksrelation} zeigt ein Beispiel für eine Gleichung. \lipsum

\lipsum[27]

{
\renewcommand{\arraystretch}{1.2}% grßerer Zeilenabstand
\sisetup{range-phrase=~{--}~}% Gedankenstrich statt "bis" bei SIrange
\begin{table}[H]
	\begin{center}
		\caption{Emissionsfaktoren von Biogasanlagen mit direkter Biogasverbrennung}
		\begin{tabu} to \textwidth {X X[1.5, r]}
			\hline
			Schadstoff	& Emissionen in \si[per-mode=symbol]{\mgkwh}					\\ \hline
			\ce{CO}		& \SIrange{922}{1116}{\relax}                               	\\
			\ce{SO2}	& \SI{90}{\relax}                                       		\\
			\ce{NO_X}	& \SIrange{727}{1944}{\relax}                               	\\
			\ce{NMVOC}	& \SIrange{36}{76}{\relax}                                  	\\
			\ce{CH2O}	& \SIrange{31}{50}{\relax}                                  	\\ \hline
			\multicolumn{2}{l}{Quellen: \cite{Paolini2018}}
		\end{tabu}
		\label{tab:tab_air-pollutants}
	\end{center}
	\vspace{-3mm}%Put here to reduce too much white space after your table
\end{table}
}

\autoref{tab:tab_air-pollutants} zeigt ein Beispiel für eine Tabelle. \lipsum

\lipsum[14]

\begin{code}
    \captionof{listing}{Gaußsches Eliminationsverfahren}\label{code:GaussElem}
    \begin{minted}[
		linenos=true,
		numbersep=-10pt,
		obeytabs=true,
		tabsize=4,
		]{matlab}
		%% Gaussian Elimination
		% bringing the Matrix into upper triangular form
		% initializing a new matrix to keep the previous results
		GaussMat = OrdMat;
		% and a new vector
		GaussVec = OrdVec;
		% for loop over m-1 elements
		for i = 1:m-1
		   % Calculating the Row Multiplier by dividing the
		   % i Element of the row i+1:m through the
		   % i,i Element of the matrix
		   Mult = GaussMat(i+1:m,i) / GaussMat(i,i);
		   % Calculating the new rows of the matrix
		   GaussMat(i+1:m,:) = GaussMat(i+1:m,:) - Mult*GaussMat(i,:);
		   % and the vector
		   GaussVec(i+1:m,:) = GaussVec(i+1:m,:) - Mult*GaussVec(i,:);
		end

		% removing rounding errors near zero
		threshold=10^(-12);
		% set near zeros to zero in the Matrix
		GaussMat(abs(GaussMat)<threshold) = 0;
		% set near zeros to zero in the Vector
		GaussVec(abs(GaussVec)<threshold) = 0;
		% initializing the solution vector x
		X = zeros(m,1);
		% determining the first element of the solution
		X(m,:) = GaussVec(m,:)/GaussMat(m,m);
		% for loop over the missing solutions
		for i = m-1:-1:1
		   % Calculating the i element of x by substracting product
		   % of the known elements of x and the corresponding row of
		   % the Gauss Matrix from the corresponding element of the
		   % Gauss Vector and dividing the outcome by
		   % the matrix element i,i
		   X(i,:) = ((GaussVec(i,:) -...
					GaussMat(i,i+1:m)*X(i+1:m,:))/GaussMat(i,i));
		end
	\end{minted}
\end{code}

Im \autoref{code:GaussElem} findet sich ein Beispiel für die Darstellung eines Codeblockes. \lipsum

\lipsum[18] Beispiel für die Darstellung von \texttt{SI}-Einheiten: \SIrange{1.20}{7.32}{\newton\per\square\meter}. \lipsum

Hier findet sich ein Beispiel für eine Abkürzung: \gls{FEE}. Anschließend wird automatisch die Kurzform ausgegeben: \gls{FEE}. Um die Darstellung der Liste zu überprüfen, findet sich hier eine weitere Abkürzung im Plural: \glspl{EEG}. \lipsum~Ein weiteres Beispiel für eine Quelle mit Seitenzahl: \cite[][vgl. S. 12]{WIKUE2006}