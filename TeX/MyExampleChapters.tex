\newpage
\lipsum[24]

\input{MySVGExample}

In \autoref{fig:SurfPlot} findet sich ein Beispiel für eine \texttt{SVG}-Grafik. \lipsum[13]

\lipsum[32]

\begin{equation}
	E = \hbar \omega
	\label{eq:plancksrelation}
\end{equation}

\autoref{eq:plancksrelation} zeigt ein Beispiel für eine Gleichung. \lipsum

\lipsum[27]

\input{MyTabuExample}

\autoref{tab:tab_air-pollutants} zeigt ein Beispiel für eine Tabelle. \lipsum

\lipsum[14]

\input{MyCodeExample}

Im \autoref{code:GaussElem} findet sich ein Beispiel für die Darstellung eines Codeblockes. \lipsum

\lipsum[18] Beispiel für die Darstellung von \texttt{SI}-Einheiten: \SIrange{1.20}{7.32}{\newton\per\square\meter}. \lipsum

Hier findet sich ein Beispiel für eine Abkürzung: \gls{FEE}. Anschließend wird automatisch die Kurzform ausgegeben: \gls{FEE}. Um die Darstellung der Liste zu überprüfen, findet sich hier eine weitere Abkürzung im Plural: \glspl{EEG}. \lipsum~Ein weiteres Beispiel für eine Quelle mit Seitenzahl: \cite[][vgl. S. 12]{WIKUE2006}